\documentclass{article}
\usepackage{bkc}
\usepackage{ccfonts}
\usepackage{tikz-cd}

\setassn{Math 6510 Problem Set 7}

\usetikzlibrary{decorations.markings,decorations.pathmorphing,matrix,arrows}
\tikzset{middlearrow/.style={
        decoration={markings,
            mark= at position 0.5 with {\arrow{#1}} ,
        },
        postaction={decorate}
    }
}

\newcommand{\im}{\text{im }}

\begin{document}
\begin{exercise}{2.2.1}{\parindent}
  Prove the Brouwer fixed point theorem for maps $f: D^n \to D^n$ by
  applying degree theory to the map $S^n \to S^n$ that sends both the
  northern and southern hemispheres of $S^n$ to the southern
  hemisphere via $f$.
\end{exercise}
\begin{solution}{\parindent}
  Suppose that there were such a map $f: D^n \to D^n$ such that $f(x)
  \neq x$ for every $x \in D^n$. Then there is a homotopy
  \[
  H(x,t) = \frac{f(x)(1-t) + x(t)}{\norm{f(x)(1-t) + x(t)}}
  \]
  which is well-defined because $f$ fixes no point. This homotopy in
  particular defines a retraction $r: D^n \to S^{n-1}$ of the disk
  onto its boundary sphere. Let $i$ be the inclusion $S^{n-1}\to D^n$
  so that $r\circ i = \mathbb{1}$. This implies that the induced
  homomorphism $r_\ast i_\ast = \mathbb{1}$ in the (reduced)
  homology. However, we know that $\tilde{H}_{n-1}(S^{n-1}) \cong Z$
  because and that $\tilde{H}_{n-1}(D^n) \cong 0$. This implies an
  isomorphism $\varphi: 0 \to \Z$ which is clearly impossible. This
  contradiction establishes that no such $f$ exists.
\end{solution}

\begin{exercise}{2.2.2}{\parindent}
  Given a map $f:S^{2n} \to S^{2n}$, show that there is some point $x
  \in S^{2n}$ with either $f(x) = x$ or $f(x) = -x$. Deduce that every
  map $\R P^{2n} \to \R P^{2n}$ has a fixed point. Construct maps $\R
  P^{2n-1} \to \R P^{2n-1}$ without fixed points from linear
  transformations $\R^{2n} \to \R^{2n}$ without eigenvectors.
\end{exercise}
\begin{solution}{\parindent}
  For the first claim, suppose that $f$ and $-f$ both have no
  fixed-points. This implies that $\deg(f) = (-1)^{2n+1} = -1$. But
  $\deg(-f) = (-1)^{2n+1} = -1$ and so $\deg(f) = 1$, a
  contradiction. Thus, either $f$ or $-f$ has a fixed point and so
  there is a point $x \in S^{2n}$ with either $f(x) = x$ or $f(x) =
  -x$.

  Now let $g: \R P^{2n} \to \R P^{2n}$. Then $g$ lifts to a map
  $\tilde{g}: S^{2n} \to S^{2n}$ and so we can find a point
  $\tilde{x}$ such that $\tilde{g}(\tilde{x}) = \pm x$. Then passing
  back down to the quotient via that projection $p$, we see that
  $p(x)$ must be a fixed point for $g$.

  For the last part of the problem, consider a non-identity
  automorphism $T \in \text{Aut}(\R^{2n})$. Then $T$ has no fixed
  points. We then construct a map $f = p \circ T \circ p^{-1}: \R
  P^{2n-1} \to \R P^{2n-1}$ with no fixed points by lifting to
  $S^{2n}$, apply $T$ and then composing with the projection map. This
  has no fixed points because otherwise we would have $(p \circ T
  \circ p^{-1})(x) = x$ which implies that $T(p^{-1}(x)) =
  p^{-1}(x)$. This is impossible becuase $T$ has no fixed points.
\end{solution}

\begin{exercise}{2.2.3}{\parindent}
  Let $f: S^n \to S^n$ be a map of degree zero. Show that there exist
  points $x,y \in S^n$ with $f(x) = x$ and $f(y) = -y$. Use this to
  show that if $F$ is a continuous vector field defined on the unit
  ball $D^n$ in $\R^n$ such that $F(x) \neq 0$ for all $x$, then there
  exists a point on $\partial D^n$ where $F$ points radially outward
  and another point on $\partial D^n$ where $F$ points radially
  inward.
\end{exercise}
\begin{solution}{\parindent}
  Suppose towards a contradiction that $f(x) \neq x$ for each $x \in
  S^n$. Then $\deg (f) = (-1)^{n+1}$. Likewise, if $f(y) \neq y$ for
  each $y \in S^n$, then the line from $y$ to $f(y)$ does not pass
  through the origin. Given this fact, consider the homotopy
  \[
  H: (y,t) \mapsto \frac{(1-t)f(y) + ty}{\norm{(1-t)f(y) + ty}}
  \]
  This is well-defined because the denominator is non-zero by
  assumption. Moreover, we see that $H(y,0) = f(y)$ and $H(y,1) =
  \mathbb{1}$. So then $\deg(f) = \deg(\mathbb{1}) = 1$, which is
  non-zero. Therefore, $f$ must have a fixed point. Likewise, $-f$
  must also have a fixed point. The contradiction establishes the
  first part.

  Now let $F$ be a non-zero vector field on $D^n$. First, define the
  map $\alpha: D^n \to S^{n-1}$ given by $\alpha(x) =
  F(z)/\norm{F(z)}$. Then consider the restriction of $\alpha$ to
  $\partial D^n$ given by $\alpha\mid_{\partial D^n}: S^{n-1} \to
  S^{n-1}$. We see that the restriction $\alpha\mid_{\partial D^n}$
  factors through $D^n$, which is contractible, and therefore must be
  nullhomotopic. Then $\deg (\alpha\mid_{\partial D^n}) = 0$ and so by
  the above we have points $x,y \in \partial D^n$ such that
  $\alpha\mid_{\partial D^n}(x) = x$ and $\alpha\mid_{\partial D^n}(y)
  = -y$. Then by the definition of $\alpha$ we see that $F(x) =
  x\norm{F(x)}$ and $F(y) = -y\norm{F(y)}$, which are vectors pointing
  radially outward and inward, respectively.
\end{solution}

\begin{exercise}{2.2.4}{\parindent}
  Construct a surjective map $S^n \to S^n$ of degree zero, for each $n
  \geq 1$.
\end{exercise}
\begin{solution}{\parindent}
  First consider the canonical injection $i:S^n \to D^n$ by sending
  $S^n$ to the boundary of $D^n$. Because $D^n$ is contractible, this
  map is nullhomotopic. We then observe that $S^n$ is the quotient
  $D^n/S^{n-1}$ and if we let $p$ be the quotient map, we see that
  $p\circ f$ is surjective, and homotopic to the constant map, and
  therefore $\deg(p\circ f) = 0$, as desired.
\end{solution}

\begin{exercise}{2.2.6}{\parindent}
  Show that every map $S^n \to S^n$ can be homotoped to have a fixed
  point if $n > 0$.
\end{exercise}
\begin{solution}{\parindent}
  Consider a map $f: S^n \to S^n$ and a point $x \in f(S^n)$. Let $y
  \in f^{-1}(x)$ let $T$ be the rotation of the sphere such that $T(x)
  = y$. Because $S^n$ is path-connected define a homotopy $H(x,t)$
  connecting the identity map $\mathbb{1}$ to $T$ by moving each point
  along a path between $x_0$ and $T(x_0)$. Then $H(x,0) = x$ and
  $H(x,1) = T(x)$. We then compose $f$ with such a homotopy and
  observe that
  \begin{align*}
    (f\circ H)(x,0) &= f(x) \\
    (f\circ H)(x,1) &= (f\circ T)(x)
  \end{align*}
  But then we see that $(f\circ T)(x) = f(y) = x$. So $f$ can be
  homotoped to a map with a fixed point.
\end{solution}

\begin{exercise}{2.2.7}{\parindent}
  For an invertible linear transformation $f: \R^n \to \R^n$ show that
  the induced map on $H_n(\R^n, \R^n - \{0\}) \approx
  \tilde{H}_{n-1}(\R^n - \{0\} \approx \Z$ us $\mathbb{1}$ or
  $-\mathbb{1}$ according to whether the determinant of $f$ is
  positive or negative.
\end{exercise}
\begin{solution}{\parindent}
  % Let $f:\R^n \to \R^n$ be an invertible linear transformation. Then
  % $f$ has a matrix representation $A_f: \R^n \to \R^n-\{0\}$. Observe
  % that because $\R^n - \{0\}$ deformation retracts onto $S^{n-1}$ we
  % have a map $r: \R^n - \{0\} \to S^{n-1}$ and let $i: S^{n-1} \to
  % \R^n-\{0\}$ be the canonical inclusion. Then because $ri \cong
  % \mathbb{1}$ and $ir \cong 1$ we see that the induced maps $r_\ast
  % i_\ast = \mathbb{1}$ and $i_\ast r_\ast = \mathbb{1}$. This leads to
  % the following diagram of homology groups
  % \begin{center}
  %   \begin{tikzcd}
  %     & & \tilde{H}_{n-1}(S^{n-1}) \arrow{d}{i_\ast} & \\
  %     H_n(\R^n) \cong 0 \arrow{r} & H_n(\R^n, \R^n-\{0\})
  %     \arrow{r}{\partial} \arrow{d}{A_\ast} &
  %     \tilde{H}_{n-1}(\R^n-\{0\}) \arrow{r} \arrow{d}{A_\ast}&
  %     \tilde{H}_{n-1}(\R^n) \cong 0 \\
  %     H_n(\R^n) \cong 0 \arrow{r} & H_n(\R^n, \R^n-\{0\})
  %     \arrow{r}{\partial} & \tilde{H}_{n-1}(\R^n-\{0\}) \arrow{r}
  %     \arrow{d}{r_\ast} &
  %     \tilde{H}_{n-1}(\R^n) \cong 0 \\
  %     & & \tilde{H}_{n-1}(S^{n-1})
  %   \end{tikzcd}
  % \end{center}
  % By the above and the fact that the long exact sequence of a pair is
  % exact, we have that the diagram commutes and has exact rows. That
  % is, the connecting homomorphisms are isomorphisms. If we topologize
  % $GL(\R^n)$ as a subspace of $\R^{n^2}$ we see that it is locally
  % path-connected and locally compact so the path-components and
  % components coincide. Thus, the map $A \mapsto \text{sign}(\det A)$
  % is continuous from $GL(\R^n) \to \{\pm 1\}$. Then there is a path in
  % $GL(\R^n)$ from $\alpha: A \to c\text{ sign}(\det A)A$ for any scalar
  % $c$. Then there is a homotopy $H(A,t) = \alpha(t)A$.

  % {\footnotesize Note: I think there is a much simpler proof of this
  %   by looking at the elementary matrices that generate $GL(\R^n)$, but
  %   the proof didn't use any of the techniques from this section.}

  Following the hint, we begin by noting that we can perform Gaussian
  elimination via a sequence of multiplcation by elementary matrices
  on the matrix representation for the map $f$. This will define a
  homotopy from $A_f$, the matrix corresponding to $f$, to a diagonal
  matrix $D$ with $\pm 1$ on the diagonals. Because $\det$ is a
  homomorphism, multiplications by suitable elementary doesn't change
  the sign of the determinant, and so $\det A = \det D$. Now we note
  that $S^{n-1}$ is a deformation retract of $\R^n - \{0\}$ and so
  because the degree of any map $S^{n-1} \to S^{n-1}$ that is a
  negation of one of the coordinates has degree $-1$ and so the same
  must hold in $\R^{n}-\{0\}$. By the long exact sequence of the pair
  we see that the same must hold in $H_n(\R^n,\R^{n}-\{0\}$. So
  $D_\ast$ will have degree $(-1)^\ell$ where $\ell$ is the number of
  $-1$'s on the diagonal. Becuase $A \sim D$ $A_\ast$ will have the
  same degree, which is precisely $\pm 1$ depending on the determinant
  of $D$.
\end{solution}

\begin{exercise}{2.2.8}{\parindent}
  A polynomial $f(z)$ with complex coefficients, viewed as a map $\C
  \to \C$, can always be extended to a continuous map of one-point
  compactifications $\hat{f}: S^2 \to S^2$. Show that the degree of
  $\hat{f}$ equals the degree of $f$ as a polynomial. Show also that
  the local degree of $\hat{f}$ at a root of $f$ is the multiplicity
  of the root.
\end{exercise}
\begin{solution}{\parindent}
  Let $f: \C \to \C$ be a polynomial. By the fundamental theorem of
  algebra $f$ admits a factorization
  \[
  f(z) = \prod_{j=1}^n (z-z_j)^{\alpha_j}
  \]
  Now note that $p:S^2 \to \C$ is a covering space and so $f$ lifts to
  a map $\tilde{f}$. Now for each root $z_j$ we can find a
  neighborhood $U_j$ containing $p^{-1}(z_j)$ and we can choose the
  $U_j$ to be disjoint in $S^2$. Now let $U_j$ is mapped to a
  neighborhood of $0$, say $V_j$. This induces a map on the relative
  homology groups
  \[
  \tilde{f}_\ast: H_2(U_j, U_j - \{\tilde{z}_j\}) \to H_2(V_j, V_j - \{0\})
  \]
  By the commutative diagram at the top of page 136, we see that each
  of the groups can be indentified with $H_2(S^2) \cong \Z$ so that
  $\tilde{f}_\ast$ is multiplication by an integer which is $\deg
  \tilde{f}\mid z_j$. Now because the neighborhoods $U_j$ are
  disjoint, we see that $\tilde{z_j}$ is the only root of $\tilde{f}$
  in $U_j$. So if we restrict $f$ to $U_j$ we get a map looking like
  \[
  f\mid_{U_j}(z) = (z-z_j)^{\alpha_j\log(z-z_j)}\prod_{i\neq
    j}(z-z_i)^{\alpha_i}
  \]
  This map is clearly $\alpha_j$-to-one onto $V_j$ because the
  exponential wraps around the circle at least $\alpha_j$ times in a
  sufficiently small deleted-neighborhood of $\tilde{z}_j$. Thus, a
  generator for $H_2(U_j,U_j-\{\tilde{z}_j\})$ maps to $\alpha_j$
  times a generator for $H_2(V_j,V_j-\{0\})$. This implies that the
  local degree of $\tilde{f}$ at $z_j$ is $\alpha_j$. By Proposition
  2.30
  \[
  \deg(f) = \sum_j \deg(f)\mid z_j = \sum_j \alpha_j
  \]
  Moreover, by the fundamental theorem of algebra we see that the
  degree of $f$ (as a polynomial) is $\sum_j \alpha_j$ as desired.
\end{solution}
\end{document}