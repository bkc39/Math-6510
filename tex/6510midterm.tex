\documentclass{article}
\usepackage{bkc}
\usepackage{ccfonts}
\setassn{Math 6510 Midterm}

\begin{document}

\begin{problem}{1}{\parindent}
  Let T be the torus $S^1 \times S^1$ and let $T'$ be $T$ with a small
  open disk removed. Let $X$ be obtained from $T$ by attaching two
  copies of $T'$, identifying their boundary circles with the
  longitude and meridian circles $S^1 \times \{x_0\}$ and $\{x_0\}
  \times S^1$ in $T$. Find a presentation for $\pi_1(X)$.
\end{problem}

\begin{solution}{\parindent}
  We will compute the fundamental group of $X$ by applying van
  Kampen's theorem. We first begin by verifying the hypotheses of the
  theorem. Consider an open neighborhood of $X$, say $U_X$ embedded in
  $\R^3$ chosen so that it deformation retracts onto $X$. Because
  deformation retraction is a homotopy equivalence, $\pi_1(U_X) \cong
  \pi_1(X)$. If we let $X_1$ be the embedding of $T$ in $U_X$ and
  $X_2,X_3$ be the respective embeddings of $T'$, then the union of
  open neighborhoods about each of them will cover $X$. Each of the
  $X_i$ is path-connected and the intersection of $X_i\cap X_j \simeq
  S^1$ for $i\neq j$, which is also path-connected. Because the
  intersection $X_1 \cap X_2 \cap X_3 = \{x_0\}$, where $x_0$ is the
  intersection of the boundary circles, the three way intersection is
  path connected as well. So we have a map
  \[
  \Phi: \pi_1(X_1) \ast \pi_1(X_2) \ast \pi_1(X_3) \lra \pi_1(X)
  \]
  that is surjective. Moreover, we see that $\pi_1(X) \cong \ast_i
  X_i/\ker(\Phi)$. So we are left to compute $\pi_(X_i)$ as well as
  impose relations for each element of $\ker(\Phi)$ and we will be
  done.

  The fundamental group of $X_1 \simeq S^1 \times S^1$ is known, $\Z
  \times \Z$. To compute the fundamental group $\pi_1(T')$ we again
  consider the square with a small disk removed. If we look at this as
  a CW-complex we have one vertex and 5 edges as in the attached
  diagram (see attached).

  We can then read off the relation on the fundamental group from the
  above by noting that it deformation retracts onto the boundary (via
  a straght line homotopy and read off the relation $\ell[e_1,e_2]$,
  where $[e_1,e_2] = e_1e_2e_1^{-1}e_2^{-1}$ is the commutator of
  $e_1$ and $e_2$. So we see that 
  \[
  \pi_1(T') \cong \lbrace e_1,e_2,e_3\mid e_3[e_1,e_2]\rbrace
  \]
  Now we note that the space $X$ could be indentified as in the
  attached picture with the identifications $\ell_1 \simeq a$ and $\ell_2
  \simeq b$. Looking at the space $X_1$, we get the relation $[a,b] =
  1$. After the identifications and the observation that $\ell_n =
  [g_n,g_{n+1}]$ (by the relation in $\pi_1(T')$ this translates to
  the relations
  \[
  [a,b] = aba^{-1}b^{-1} =
  [g_1,g_2][g_3,g_4][g_1,g_2]^{-1}[g_3,g_4]^{-1} = 1
  \]
  Note, that these commutators are the elements in $\ker(\Phi)$
  because they are all the elements of the form
  $i_{nm}(\omega)i_{mn}^{-1}(\omega)$, which is precisely
  $\ker(\Phi)$. Thus, we have the computation
  \[
  \pi_1(X) \cong \lbrace g_1,g_2,g_3,g_4 \mid
  [g_1,g_2][g_3,g_4][g_1,g_2]^{-1}[g_3,g_4]^{-1} \rbrace
  \]

\end{solution}

\begin{problem}{2}{\parindent}
  Consider the quotient space of a cube $I^3$ obtained by identifying
  each square face with the opposite square face via the right-handed
  screw motion consisting of a translation by one unit in the
  direction perpindicular to the face combined with a one-quarter
  twist of the face about its center point. Show this quotient space
  $X$ is a CW complex with two 0-cells, four 1-cells, three 2-cells,
  and one 3-cell. Using this structure, show that $\pi_1(X)$ is th
  quaternion group $\{\pm 1, \pm i, \pm j, \pm k\}$, of order eight.
\end{problem}

\begin{solution}{\parindent}
  Consider the cube identified as in the attached diagram: We then do
  the identification via the ``screw motion''. and get the quotient
  space $X$, (in diagram).

  To see the CW-complex structure we see there are two 1-cells, one
  for vertices $v_0,v_2,v_5,v_7$ another for the vertices
  $v_1,v_3,v_4,v_6$. The 2-cells would correspond to the opposite
  faces of the original cube (th three classes are shown in the
  attached diagram.. Lastly, the 3-cell would attach along each face
  and ``fill'' the space.

  Now we will compute the fundamental group, $\pi_1(X)$. Let us look
  at the 2-cells, these give the relations $abcd, c^{-1}d^{-1}ba,
  adb^{-1}c^{-1}$ (all of these products are the identity, this will
  be used in the computations). In addition, we showed that attaching
  a $k$-cell for $k > 2$ does not affect the computation of
  $\pi_1(X)$, the the one 3-cell in the construction is irrelevant.

  We need to find the isomorphism between $\pi_1(X)$ and $Q_8$. We
  have a familiar presentation for $Q_8$ (c.f. \textit{Abstract
    Algebra} by Dummit and Foote) given by
  \[
  Q_8 = \lbrace x,y | x^4, y^{-2}x^2, xyxy^{-1}\rbrace
  \]
  We need to find how the relations above for $\pi_1$ are equivalent
  to these. We begin by finding the elements of order two. Note that
  \[
  abab = adcb = adad
  \]
  For the element of order 4, we compute
  \[
  (ab)^4 = (cd)^{-1}(ab)^2(cd)^{-1} = (d^{-1}c^{-1})adcb(d^{-1}c^{-1})
  = d^{-1}bac^{-1} = d^{-1}dcc^{-1}
  \]
  Thus, $(ab)$ has order 4 in $\pi_1(X)$. For the last relation, we
  will show the equivalent condition $(ad)(ab)(ad)^{-1} =
  (ab)^{-1}$. Indeed, observe that
  \[
  (ad)(ab)(ad)^{-1} = (ad)abb^{-1}c^{-1} = adac^{-1} = cbac^{-1} = cd = (ab)^{-1}
  \]
  By the relations, we also see that $(cd), (cb)$ are generated by
  $(ab),(ad)$ and so $(ab),(ad)$ are the desired generators for the
  presentation of $Q_8$. So the map $(ab) \mapsto x$, $(ad) \mapsto y$
  is the desired isomorphism.
\end{solution}

\begin{problem}{3}{0cm}
  \vspace*{-\bigskipamount}
  \begin{enumerate}
  \item Show that $f: X \to Y$ is a homotopy equivalence if there
    exist maps $g,h: Y \to X$ such that both compositions $gf$ and
    $fh$ are homotopic to the identity.
  \item Let $X$ and $Y$ be path-connected and locally path-connected,
    and let $\tilde{X} \to X$ and $\tilde{Y} \to Y$ be
    simply-connected covering spaces. Show that if $X \simeq Y$ then
    $\tilde{X} \simeq \tilde{Y}$.
  \end{enumerate}
\end{problem}

\begin{solution}{0cm}
  \begin{enumerate}
  \item Suppose that $f: X \to Y$ has both left and right inverses
    $g,h: Y \to X$, respectively. Because $fh \simeq \mathbb{1}_Y$ and
    $gf \simeq \mathbb{1}_X$, we see $g \simeq gfh$ and $h \simeq
    gfh$. Applying $f$ to these relations we get that $fh \simeq
    \mathbb 1_Y \simeq fg$. Thus, $g$ is in fact a right-inverse to $f$
    so $f$ is a homotopy equivalence.
  \item Let $p_1: \tilde{X} \to X$ and $p_2: \tilde{Y} \to Y$ be the
    covering maps. We will construct a map $\tilde{f}: \tilde{X} \to
    \tilde{Y}$ that is a homotopy equivalence by precomposing the
    lifts of the homotopy equivalence $f,g$ with their respective
    covering maps. More precisely, if we have a point $\tilde{\alpha}
    \in \tilde{X}$, there is a path from $\tilde{\alpha}$ to the
    desired basepoint in $\tilde{X}$, say $\tilde{x}$. We know that
    such a path must exist because of Proposition 1.33. If we let
    $\tilde{ell}$ be the path, then we can apply $p_1$ to
    $\tilde{ell}$ and project down to a usual path in $X$ from the
    basepoint $x$ to $p{\tilde{\alpha}} = \alpha$, $\ell$. We then
    apply $f$ to map into $y$ to get a path in $Y$ that lifts
    $f\ell$. We define our desired map, $\tilde{f}$ as moving
    $\tilde{alpha}$ to the endpoint of this lifted path. Note that
    this construction says that $p_2(\tilde{f\ell}) \simeq f\ell$ and
    so $p_2 \tilde{f} \simeq fp_1$. We define the proposed homotopy
    inverse for $\tilde{h}$, call it $\tilde{g}$ analogously. Namely,
    in such a way that $p_1(\tilde{g}) \simeq gp_2$.

    We need to show that this is indeed a homotopy equivalence. If we
    consider the construction we have shown that
    $p_1\tilde{g}\tilde{f} \simeq gp_2\tilde{f} \simeq p_1$ and
    likewise $p_2\tilde{f}\tilde{g} \simeq fp_1\tilde{g} \simeq
    p_2$. These equivalences follow from the homotopy lifting property
    because we can simply restrict the homotopy of $fg$ and $gf$ to
    the respective identities, and by the uniqueness this lift must be
    $\tilde{f}\tilde{g}$ and $\tilde{g}\tilde{f}$, respectively.
    Thus, this is in fact a homotopy equivalence and we are done.
  \end{enumerate}
\end{solution}

\begin{problem}{4}{0cm}
  \vspace*{-\bigskipamount}
  \begin{enumerate}
  \item Using covering spaces, enumerate all the different subgroups
    of $\Z \ast \Z$ of index 3 and determine which of these are normal
    subgroups. Your answer should include some justification why this
    list is complete.
  \item For $p$ prime, find all the index $p$ normal subgroups of $\Z
    \ast \Z$ and the corresponding covering spaces of $S^1 \vee S^1$.
  \end{enumerate}
\end{problem}

\begin{solution}{0cm}
  \begin{enumerate}
  \item Let $X = S^1 \vee S^1$ so $\pi_1(X) \cong \Z \ast \Z$. We
    then apply Proposition 1.36 from the book to see that every
    subgroup of $G \leq \Z \ast \Z$ has a corresponding covering space
    $p: \tilde{X}_G \to X$. Because the subgroups we are looking for
    have index 3, we apply Theorem 1.38 to see that the corresponding
    covering spaces must be the connected 3-sheeted covering spaces of
    $S^1 \vee S^1$. 

    We know that each covering space must be regular of degree 4 with
    in-degree 2 and out-degree 2, because the covering space must be
    locally homeomorphic to the base space. Moreover, because each
    edge must be the lift of a generating loop in the base space, we
    see that there must be one ``in'' edge and one ``out'' edge for
    each generator in the base space. In particular, this means that
    a self-loop in the covering space forces the other edges to go to
    a different vertex because otherwise the graph would be
    disconnected. As a result, we can characterize the graph by the
    number of vertices with self-loops. We consider each case:
    \begin{enumerate}
    \item If there are no self loops, then there is only one graph
      possible (up to isomorphism) because of the relations on the
      edges. We get the graph attached.

    \item If there is one self loop then we can see that there is only
      one possible graph satsifying the restriction on the edges.
    \item If there are two loops we also have only one possibility.
    \item Finally, in the case that there are three self loops, there
      can only be one possible choice.
    \end{enumerate}
    We can see that only two of the covering spaces are regular: the
    one with no self-loops and the one with three self-loops. The
    intuition is clear: the other two do not have maximal
    symmetry. Phrasing this in the language of deck transformations,
    we see that there must be an automorphism of the covering space
    that takes a given vertex, say $v_0$, to another vertex,
    $v_1$. This is clearly impossible, because any graph automorphism
    would preserve self loops, so there is no autmorphism that could
    take a vertex with a self-loop to one without one. So all of the
    vertices must have the same number of self loops, either one or
    zero. 
  \item The critical fact is the subgroups we are looking for are
    normal. By the same reasoning as the previous part, we are looking
    for connected $p$-sheeted covering spaces of $S^1 \vee S^1$. By
    proposition 1.39 we see that normal covering spaces are in
    bijective correspondence with normal subroups. Moreover, we knkow
    that $G \cong \pi_1(X)/H$, where $H$ is the given normal
    subgroup. Because $G \cong \Z \ast \Z$ this implies that there is
    a surjective homomorphism $\varphi: \Z \ast \Z \to
    \Z/p\Z$. Because homomorphisms are determined by their values on
    the generators, we can specify all possible homomorphisms by
    considering where it would send the generating loops of the base
    space.

    Because $\varphi$ is surjective, we see that it cannot send both
    of the generating loops to 0. There are then two cases to consider
    \begin{enumerate}
    \item If $\varphi$ sends one generator to zero then it must send
      the other generator a non-zero element, which has order
      $p$. This leads to a covering space that has one $p$-cycle and
      every vertex has a self-loop, corresponding to the generator
      mapped to zero.
    \item Otherwise $\varphi$ sends both generators to elements of
      order $p$ in $\Z/p\Z$. This leads to a graph that has two $p$
      cycles, one for each generator.
    \end{enumerate}
    This is all of the covering spaces, and therefore subgroups, up to
    isomorphism. To count them the first case gives us one possible
    covering space up to isomorphism: the graph with $p$
    self-loops. The second case only contributes one addition covering
    space (up to isomorphism) because we can simply relabel the
    vertices to have the two $p$-cycles map to one another (they are
    conjugates in $\Z \ast \Z$).
  \end{enumerate}
\end{solution}

\begin{problem}{5}{\parindent}
  \medskip
  Let $T: \R^2 \to \R^2$ be the linear transformation $T(x,y) =
  (2x,y/2)$. This generates an action of $\Z$ on $X = \R^2 - \{0\}$.
  \begin{enumerate}
  \item Show this action of $\Z$ on $X$ is a covering space action.
  \item Show that $X/\Z$ is not a Hausdorff space and describe how it
    is the union of four subspaces homeomorphic to $S^1 \times \R$,
    two coming from the complementary components of the $x$-axis and
    two coming from the complimentary components of the $y$-axis.
  \item Show that $\pi_1(X/\Z) \cong \Z \times \Z$ and construct a map
    $S^1 \times S^1 \to X/\Z$ that induces an isomorphism on $\pi_1$
    and all higher homotopy groups $\pi_n$.
  \end{enumerate}
\end{problem}

\begin{solution}{0cm}
  Let the action of $\Z$ on $X$ be given by $n \cdot v = T^{n}(v)$.
  \begin{enumerate}
  \item We need to find a neighborhood about a point $(x,y)$ such that
    $T^n(U) \cap T^m(U) = \emptyset$ when $m \neq n$. Consider first
    positive values of $m$ and $n$ and suppose that $m > n$. Then the
    radius $r$ must satisfy the relation
    \[
    |2^m(x -r) - 2^n(x+r)| > 0
    \]
    Solving for $r$ we see that $r < (2^m-2^n)x/(2^m+2^n)$. Letting
    $m,n \to \infty$ sends $r$ to zero, so the largest difference
    possible is when $n=1,m=2$ and we get $r < x/3$. For negative
    values of $n,m$ we apply the same inequality in the $y$-coordinate
    to see that $r < y/3$. If $n$ and $m$ have different signs then
    the sets $T^m(U)$ and $T^n(U)$ are clearly disjoint by applying
    the inequality with $-n$ replaced for $n$. Thus, a ball of radius
    $\min\{x/3, y/3\}$ is the desired neighborhood.
  \item First we will show that the orbit space $X/\Z$ is not
    Hausdorff. Choose two points $(x_0,y_0), (x_1,y_1)$ in $X$ and two
    open neighborhoods $U_0, U_1$ about each point respectively. we
    must show that $U_1 \cap U_2 \neq \emptyset$ in the orbit
    space. Because $U_0, U_1$ are bounded we see that eventually
    $T^n(U_i)$ will intersect the $x$-axis for large enough values of
    $N$ (because $y/2^n \to 0$ for any $y$ as $n \to \infty$. The
    intersections will be intervals on the axis, both of which are
    bounded. Suppose that the first interval is $I_0 = \alpha_0 \pm
    \epsilon_0$ and the second is $I_1=\alpha_1 \pm \epsilon_1$. We
    need to find a point that is in the orbit of both of these
    intervals. Suppose that $I_0$ is bounded by $2^{N_0}$ and $I_1$ is
    bounded by $2^{N_1}$ and that $N_0 \leq N_1$. Then a point in the
    intersection will exist if we can satisfy
    \[
    2^{-N_0}(\alpha_0 + \epsilon_0) \geq 2^{-N_1}(\alpha_1 - \epsilon_1)
    \]
    Or equivalently
    \[
    \frac{\alpha_0 + \epsilon_0}{\alpha_1 - \epsilon_1} \geq 2^{N_0 - N_1}
    \]
    This is clearly possible for $m$ sufficiently large. Thus, any two
    neighborhoods have non-empty intersection in the quotient
    space. So the space $X/\Z$ is not Hausdorff.

    Now let's see how $X/\Z$ is homeomorphic to the union of four
    cylinders. Consider the lines $y = \pm x$ in the plane. The orbit
    space of point in $X$ lies on a unique hyperbola and we can index
    the hyperbola's by the coordinate that the intersect $y=x$. In a
    similar manner to the construction from the projective plane (the
    space of lines in the plane), we will construct the space
    $X/\Z$. Note that the analog of the stereographic projection is
    given by the $x$-coordinate of the intersection point with $y=x$
    and the hyperbolic angle $u$. There are only 4 points at
    infinity in this construction, two for each axis, and as a result,
    each hyperbola in $X$ wraps around to a loops in $X/\Z$. Thus, we
    see a cylinder for each complementary piece of the axis. if we let
    $x_v$ be the $x$-coordinate of the point where the hyperbola
    containing $v$ in $\R^2$ intersects $y = \pm x$ then there is a
    natural mapping $v \mapsto (x_v, u)$ in $X/\Z$.
  \item Now we need to show that $\pi_1(X/\Z) \cong \Z \times \Z$. We
    will do this via a map $\varphi: S^1 \times S^1 \to X/\Z$ that
    induces an isomorphism. Let the map be given by $(\theta, \psi)
    \mapsto (\cosh\theta, \sinh\psi)$. The induced map will map the
    generating circles in the torus onto the image of the deleted disk
    in $X$ and then the second angle will measure which point on the
    circle corresponding to the image of a hyperbola under the
    quotient map. In particular, this map has trivial kernel and
    therefore this induces an isomorphism on $pi_1$. To get the
    isomorphism on the higher homotopy groups because it is a covering
    space projection and so we can apply Proposition 4.1 to get that
    it induces an isomorphism on the higher homotopy groups $\pi_n$ as
    well.
  \end{enumerate}
\end{solution}
\end{document}
