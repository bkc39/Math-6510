\documentclass{article}
\usepackage[tmargin=1in,bmargin=1in,lmargin=1.5in,rmargin=1.5in]{geometry}
\usepackage{amsfonts,amsmath,amssymb,amsthm}
\usepackage{mathrsfs}
\usepackage{ccfonts}
\usepackage{relsize,fancyhdr,parskip}
\usepackage{graphicx}

\usepackage{tikz} \usetikzlibrary{shapes, shapes.geometric,
  shapes.symbols, shapes.arrows, shapes.multipart, shapes.callouts,
  shapes.misc,decorations.markings,decorations.shapes}

\pagestyle{fancy}
\lhead{Ben Carriel}
\chead{Math 6510 Problem Set 3}
\rhead{\today}

\parskip 7.2pt
\parindent 8pt

\newcommand{\tab}{\hspace*{2em}}
\newcommand{\tand}{\tab\text{and}\tab}

\DeclareMathOperator{\N}{\mathbb{N}}
\DeclareMathOperator{\Z}{\mathbb{Z}}
\DeclareMathOperator{\Q}{\mathbb{Q}}
\DeclareMathOperator{\R}{\mathbb{R}}
\DeclareMathOperator{\C}{\mathbb{C}}
\DeclareMathOperator{\D}{\mathbb{D}}
\DeclareMathOperator{\T}{\mathbb{T}}
\DeclareMathOperator{\capchi}{\raisebox{2pt}{$\mathlarger{\mathlarger{\chi}}$}}

%\DeclareMathOperator{\divides}{\mathrel{|}}
\DeclareMathOperator{\suchthat}{\mathrel{:}}

\DeclareMathOperator{\lra}{\longrightarrow}
\DeclareMathOperator{\into}{\hookrightarrow}
\DeclareMathOperator{\onto}{\twoheadrightarrow}
\DeclareMathOperator{\bijection}{\leftrightarrow}
\DeclareMathOperator{\lap}{\bigtriangleup}

\newcommand{\problem}[1]{\noindent{\textbf{Problem #1}}\\}
\newcommand{\problempart}[1]{\noindent{\textbf{(#1)}}}
\newcommand{\exercise}[1]{\noindent{\textbf{Exercise #1:}}}

\newcommand{\der}[2]{\frac{\partial #1}{\partial #2}}
\newcommand{\norm}[1]{\|#1\|}
\newcommand{\diam}[1]{\text{diam}(#1)}
\newcommand{\seq}[2]{\{#1_{#2}\}_{#2 = 1}^\infty}

\newcommand{\conj}[1]{\overline{#1}}
\newcommand{\cis}[1]{\operatorname{cis}#1}
\newcommand{\res}{\operatorname{res}}


\newcommand{\real}{\mathrel{\text{Re}}}
\newcommand{\imag}{\mathrel{\text{Im}}}
\newtheorem*{thm}{\\ Theorem}
\newtheorem*{lem}{\\ Lemma}
\newtheorem*{claim}{\\ Claim}
\newtheorem*{defn}{\\ Definition}
\newtheorem*{prop}{\\ Proposition}

\begin{document}
\exercise{1.2.2}

Suppose that $X \subset \R^n$ is the union of convex open sets,
$X_1,\ldots,X_n$ such that $X_i \cap X_j\cap X_k \neq \emptyset$ for
every $i,j,k$. We will first establish that $X$ is path-connected. To
see this, we use the hypothesis that $X_i \cap X_j\cap X_k \neq
\emptyset$ and note that this implies that the pairwise intersections
$X_i\cap X_j \neq \emptyset$ because otherwise we would have $X_i \cap
X_j\cap X_k \subset \emptyset$, but the former is non-empty so this is
impossible. With this fact in hand, we choose any two points $x,y$ in
$X$ and note that if $x \in X_i$ and $y \in X_j$ then there is a point
$z$ in $X_i \cap X_j$ so we can draw the line between $x$ and $z$ and
then from $z$ to $y$, both of which are contained in $X$ because each
of the $X_j$ is convex. Hence, $X$ is path-connected.

Now we need to compute the fundamental group of $X$. First we note
that the intersection of two convex sets must be convex because any
two points $x,y$ in the intersection $X_i \cap X_j$ is in $X_i$ and
$X_j$ and hence the line joining them is also in $X_i\cap X_j$ by
convexity. We can then note that because convex sets deformation
retract to a point via the nullhomotopy, their fundamental groups are
trivial. We then apply the Van-Kampen theorem to see that $\pi_1(X)
\cong \mathlarger{\ast}_{i,j,k} \pi_1(X_i \cap X_j \cap X_k) = \ast 0
= 0$. So the fundamental group of $X$ is trivial and $X$ is simply
connected.

\exercise{1.2.3}

As a base case, we refer to the fact that $\R^n - \{x\} \cong S^{n-1}
\times \R$ (c.f. pg 35) and note that $\pi_1(\R^n - \{x\}) \cong
\pi_1(S^{n-1}) \times \pi_1(\R)$. We then use the fact that $S^n$ is
simply connected for $n \geq 2$ so that $\pi_1(\R - \{x\}) = 0$ when
$n \geq 3$. We then proceed by induction on the number of removed
points to prove the result. Indeed we note that
\[
\R^n - \{x_1, \ldots, x_n\} \cong (\R^n - \{x_1,\ldots,x_{n-1}\})
\times (\R^n - \{x_n\})/\{x_1,\ldots,x_n\}^c
\]
This quotient identifies all points that are not one of the
$x_i$. Geometrically, it is a superposition of the two versions of
$\R^n$, one with $n-1$ points removed, and the other with 1 point
removed. Then we note that by the induction hypothesis $\pi_1(\R^n -
\{x_1,\ldots,x_{n-1}\} = 0$ and $\pi_1(\R^n - \{x_n\}) = 0$ as seen in
the base case. Consequently, the inclusion map $\R^n -
\{x_1,\ldots,x_n\} \into (\R^n - \{x_1,\ldots,x_{n-1}\}) \times (\R^n
- \{x_n\})$ induces an injective homomorphism from $\pi_1(\R^n -
\{x_1,\ldots,x_n\} \to 0$. This implies that $\pi_1(\R^n -
\{x_1,\ldots,x_n\} = 0$ as desired.

\exercise{1.2.4}

Now we consider the space $X$ which is $\bigcup_{j=1}^n
\mathcal{L}_j$, where $\mathcal{L}_j$ is a line through the
origin. First we note that $\R^3 - X$ deformation retracts onto $S^{2}
- \{\ell_1,\ldots,\ell_{2n}$, where each of the $\ell_j$ is one of the
points where the line $\mathcal{L}_j$ intersects $S^2$, via the
straight line homotopy (note that the origin is deleted so we retain
continuity). Then we note that $S^2 - \{x\} \cong \R^2$ via
stereographic projection, and this induces a map from $S^2 -
\{\ell_1,\ldots,\ell_{2n-1}\} \into \R^2$. We then note that $\R^2 -
\{\ell_1,\ldots,\ell_{2n-1}\}$ deformation retracts onto the loops
around each of the deleted points, which is $\bigvee_{i=1}^{2n-1}
C_i$. This gives a string of isomorphisms
\[
\pi_1(\R^3 - X) \cong \pi_1(S^2 - \{\ell_1,\ldots,\ell_{2n}\}) \cong
\pi_1(\R^2 - \{\ell_1,\ldots,\ell_{2n-1}\}) \cong
\pi_1(\bigvee_{i=1}^{2n-1} C_i) \cong \ast_{i=1}^{2n-1} \Z
\]

\exercise{1.2.7}

Let $X$ be the space $S^2/(n \sim s)$ where $n,s$ are the north and
sout poles, respectively. Following the hint, we will create a
cell-structure for $X$. We begin with 0-cells that are simply the
points $n,s$. We then attach a 1-cell as a path connecting $n$ to
$s$. Finally, we add a 2-cell along the path between $n$ and $s$ as
well as its inverse path. Passing to the quotient space we are left
with a cell structure for $X$ that has the generating loop as the path
from $n$ to $s$, which are now identified. Also, the two cell is
attached along the path and its inverse, so the presentation for the
group is
\[
\pi_1(X) \cong \langle a \mid aa^{-1} = 1\rangle \cong \Z
\]
So the fundamental group of $X$ is $\Z$.

\exercise{1.2.8}

We begin with two separate tori $X = (S^1 \times S^1) \times (S^1 \times
S^1)$. Let $a,b$ and $c,d$ be the generators for their respective
fundamental groups. Because
\[
\pi_1(X) \cong \pi_1(S^1 \times S^1) \times \pi_1(S^1 \times S^1)
\]
we conclude that $a$ and $b$ must both commute with $c$ and $d$,
respectively. Then, under the identification $S^1\times \{x_0\} \sim
S^1 \times \{x_1\}$ where each is a subspace of the distinct factors
of $X$, we note that this corresponds to identifying one of the
generators. Without loss of generality, suppose that we identify
$b,c$. Let $Y$ denote this quotient space, then have a presentation
for the group
\[
\pi_1(Y) = \langle a,b,d \mid ad = da\rangle
\]

\exercise{1.2.9}

First we want to show that $M_h'$ does not retract onto its boundary
circle, $C$, and that consequently $M_g$ does not deformation retract
onto $C$. Suppose to the contrary that there were such a retraction
$r: M_h' \to C$ and let $i: C \into M_h'$ be the inclusion
map. Because $r$ is a retraction the induced homomorphism $i_*:
\pi_1(C) \into \pi_1(M_h')$ is injective. Consider the abelianizations
of $\pi_1(C)$ and $\pi_1(M_h')$ and the map
\[
i_*^{ab}: \pi_1(C) \into \pi_1^{ab}(M_h')
\]
This map is well defined because the homomorphism must send the
commutator subgroup into the commutator subgroup. Also, because
$\pi_1(C) \cong \Z$ is already abelian, taking the abelianization does
not change it. Consider the identification of $M_h'$ with a regular
$4g$-gon in the usual way (c.f. pg 5). Then if we delete a point in
the center of the polygon the loop around the boundary, which consists
only of commutators, will be abelianized away. Thus, $i_*^{ab}$ is the
trivial map, which is impossible. Thus, there could be no such
retraction.

Now we need to show that $M_g$ does retract onto the non-separating
circle $C'$. We see that $C'$ can be identified with one of the
generators of $\pi_1(M_g)$, say $c$. If we identity $M_g$ with its
$4g$-gon, then we can see that the polygon is homeomorphic to a
rectangle with sides $a,b,a^{-1},[M_g,M_g]cb^{-1}$, where $a,b$ are
generators for $\pi_1(M_g)$ and $[M_g,M_g]$ is the commutator subgroup
of $\pi_1(M_g)$. The retract on this rectangle is the canonical
projection onto the sides labeled $a$.

\exercise{1.2.10}

Consider the complement of the loops $\alpha \cup \beta$ as seen in
the figure on pg 53, denote this space by $X$.. We note that the
ambient space $D^2 \times I$ has the boundary disks $S^1 \times \{0\}$
and $S^1\times\{1\}$ as contractible subspaces. We apply the
contraction and note that the image of $\alpha$ and $\beta$ is two
interlocked loops rooted at the points that we contracted onto at the
boundary. Then we note that the resulting figure, which is the union
of two cones, is homotopy equivalent to $S^2$ with interlocking
loops. It is shown in example 1.23 (c.f. pg 46) that the fundamental
group of the resulting space $Y$ is $\pi_1(Y) \cong \Z \times
\Z$. Thus, the retractions would induce injective homomorphisms into
$\pi_1(Y) \cong \Z \times \Z \into \pi_1(X)$. But because $\gamma$ is
a non-trivial loop in $Y$, it cannot be sent to a trivial loop in
$X$. So $\gamma$ is not nullhomotopic in $X$.

\exercise{1.2.17}

Let $X = \R^2 - \Q^2$. We will first show that $X$ is path-connected,
and so the computation of $\pi_1(X)$ is independent of the choice of
basepoint. Consider $(x,y),(z,w) \in X$. By construction we know that
one of $x$ and $y$ is irrational, and likewise for $z$ and $w$. We
proceed by case analysis. Suppose that both $x$ and $w$ are irrational
then the line $t(x,y) + (1-t)(x,w)$ is a traverses from $(x,y) \to
(x,w)$. Then we traverse the line $t(x,w) + (1-t)(z,w)$ is a valid
path. If we see that $z$ is irrational, then we can find some point
$(z,w')$ with $w'$ irrational and traverse this line. We are then in
the previous case, and construct the path as we did there. The case
that $y$ is irrational is analogous, and so $X$ is path connected.

Let $p$ be some irrational number and for each irrational $\xi$
consider the strips $([-\xi,\xi] \times \{p\}), ([-\xi,\xi] \times
\{-p\}), (\{p\} \times [-\xi,\xi]), (\{-p\} \times [-\xi,\xi])$. None
of these lines contain any rational points, so they form a square in
$X$. We then consider the basepoint $(0,p)$ and consider the loops
that traverse each of the ``squares'' above in the usual
orientation. We need to show that for distinct $\xi$, the loops are
not homotopic. Let $S_{\xi_1},S_{\xi_2}$ be distinct squares above and
consider the path which first traverses around $S_{\xi_1}$ and then
$S_{\xi_2}$ in the reverse direction. We need to show that this is not
the constant loop.

Because $\Q$ is dense in $\R$ we know that there is a rational $r$
between $\xi_1$ and $\xi_2$ so the inclusion map $\R^2 - \Q^2 \into
R^2 - (0,p)$, which induces a homomorphism $\pi_1(\R^2 - \Q^2) \to
\pi_1(R^2 - (0,p))$. The plane with a point removed is homotopic to
$S^1$ via the straight line homotopy, but the image of the loop
$S_{\xi_1}S_{\xi_2}^{-1}$ is a non-trivial loop in $S^1$. Hence,
$S_{\xi_1}S_{\xi_2}^{-1}$ cannot be nullhomotopic in $\R^2-\Q^2$
because otherwise it would be sent to the trivial element under the
homomorphism, which we have just seen is not the case. As a result,
there is a homotopy class in $\pi_1(\R^2-\Q^2)$ for each irrational,
and therefore $\pi_1(\R^2-\Q^2)$ is uncountable.

\exercise{A1} Let $D$ denote the disk, $A$ denote the annulus and $M$
denote the M\"{o}bius band.
\begin{enumerate}
\item[\textbf{(a)}] First consider the case of the disk. We have seen
  that if the point removed from the disk is not on the boundary, then
  $\pi_1(X) = \Z \ast \Z \not\cong 0 = \pi_1(D)$.

  In the case of the annulus, we note that if the point removed is not
  on the boundary then the annulus deformation retracts onto the two
  generating loops, one about each hole. Hence, there is an
  isomorphism $\pi_1(A - \{x\}) \cong \Z \ast \Z$. Also, we see that
  $A$ retracts onto the circle so that $\pi_1(A) \cong \Z$. However,
  the induced homomorphism would be injective from $\Z \ast \Z \into
  \Z$, which is impossible. However, if the deleted point is on the
  boundary then the resulting space still deformation retracts onto
  the inner circle and so the induced homomorphism is an isomorphism
  $\Z \to \Z$.

  Now consider the M\"{o}bius band. If we delete a point that is not
  on the boundary, then $M$ deformation retracts onto a figure-8 and
  so there should be an induced homomorphism $\pi_1(M - \{x\}) = \Z
  \ast \Z \to \Z = \pi_1(M)$, which is impossible. However, if the
  deleted point is on the boundary, then $M$ deformation retracts onto
  the remaining boundary circle and so has $\pi_1(M - \{b\}) \cong
  \Z$, so the induced homomorphism is an isomorphism.
\item[\textbf{(b)}] Let $f: X \to Y$ be a homeomorphism, where $X$ and
  $Y$ are either both disks, annuli, or M\"{o}bius bands. Suppose that
  $f$ does not restrict to a homeomorphism on the boundary so that
  $f_R = f\mid_{\partial X}$ is not a homeomorphism. Consider the
  spaces $X - \{x\}$ and $Y - \{f(x)\}$. We then apply the previous
  part to the induced isomorphism $\pi_1(X - \{x\}) \cong \pi_1(Y -
  \{f(x)\})$ such that $x \in \partial X$ and $f(x) \not\in \partial
  Y$ to reach a contradiction. These groups are only isomorphic if
  both points are contained in the boundary. Thus $f$ must restrict to
  a homeomorpshism on the boundary.
\item[\textbf{(c)}] Finally, to see that $M \not\cong A$ we suppose to
  the contrary that there were a homeomorphism $f: X to Y$. By the
  previous part, $f$ would have to restrict to a homeomorphism on the
  boundary. However, this is impossible because if $X$ is a M\"{o}bius
  band and $Y$ is an annulus then $Y$ deformation retracts onto its
  boundary circles. However, this implies that $f^1\circ r$ is a
  deformation retract of $M$. However, we showed in last week's
  assignment that $M$ does not deformation retract onto it's boundary
  circle. Thus, we have a contradiction and so $M \not\cong A$.
\end{enumerate}

\exercise{A2}

Suppose that $X\subset \R^n$ is an open, path-connected set and set $Y
= X - \{p\}$ Let $Z$ be a ball about the deleted point $p$ that is
contained in $X$. We can find such a neighborhood because $x$ is
assumed open. Observe that $X = Y \cup Z$ and note that the
intersection $Y \cap Z$ is just $Z$, a ball centered at $p$ with the
center deleted. We know that when $n \geq 3$ $Z$ is homeomorphic to
$\R^{n-1}$, which is path-connected. Hence, the intersection $Y \cap Z
\cong Z$ is path-connected and we can apply the van Kampen theorem to
see that the map $\pi_1(Y) \ast \pi_1(Z) \to \pi_1(X)$ is
surjective. However, the ball $Z$ is contractible so $\pi_1(Z) = 0$
and so in fact have that $\pi_1(Y) \to \pi_1(X)$ is surjective. We
then recall that the inclusion map also induces an injective
homomorphism $\pi_1(Y) \to \pi_1(X)$ and so the homomorphism is in
fact an isomorphism.

\exercise{A3}

Let $X \subset \R^n$ be open and path-connected. Following the hint,
we first show that any path in $X$ with rational endpoints is
homotopic to a path consisting of a finite number of line segments
with rational endpoints. Indeed, let $\gamma$ be a path in $X$ with
endpoints $p$ and $q$, which have rational endpoints. Cover each point
in $\gamma$ with a ball contained in $X$, which is always possible
because $X$ is open. Because $\gamma$ is the continuous image of the
unit interval, it is comapct, and so only finitely many of these balls
are needed to cover $\gamma$.

Let $B_1,B_2,\ldots, B_n$ be finite subcover and without loss of
generality let $p \in B_1$ and $q \in B_n$. Arrange a sequence
$B_1'B_2'B_3'\cdots B_m'$ where each $B_j'$ is one of the $B_i$ such
that $B_{j-1}' \cap B_j \neq 0$, $B_1' = B_1$ and $B_m' = B_n$ (This
sequence may contain duplicates, but will be finite). Because balls
are convex, we can construct a path from $p$ to $q$ by connecting $p$
to a rational point in the intersection $B_1' \cap B_2'$ which is
possible because the intersection is open and the rationals are
dense. This gives us a finite path from $p$ to $q$ consisting only of
line segements with rational endpoints.

To see the homotopy between $\gamma$ and our piecewise linear path we
proceed by induction on the number of balls in the subcover. If there
were only one ball, then the ball is convex so the straight line
homotopy will suffice. For the inductive step, suppose that we have a
homotopy that we have $n$ balls that cover $\gamma$. We know that
there is a homotopy for the first $n-1$ of them by the inductive
hypothesis, and we need only connect them through the intersection of
$B_{n-1}$ and $B_n$. Because the sets are convex, a piecewise homotopy
also works to connect the final two balls. (This is a shortened
version of a previous exercise).

With the hint shown, we note that any path with irrational endpoints
is homotopic to a path with rational endpoints because any ball about
the endpoints will contain a rational point by density. Because the
ball is path connected, the translation from the irrational point to
the rational one does not change homotopy classes. Thus, each path in
$X$ is homotopic to a path indexed by $\bigcup_{n=1}^\infty (\Q^n
\times \Q^n)$ which is the union over all finite length paths of all
rational endpoint line segments. This is the countable union of
countable sets. Hence there is an injection $\pi_1(X) \into \Z$ so
$\pi_1(X)$ is countable.

\end{document}